\documentclass[a4paper,xelatex,ja=standard,hiresbb,12pt]{bxjsarticle}

\usepackage{zxjatype} 

\usepackage[a4paper]{geometry}	%A4サイズ
\geometry{left=25mm,right=25mm,top=20mm,bottom=25mm} %余白
\usepackage{setspace} %行間設定
\setstretch{1.5}
\usepackage[version=4]{mhchem} %化学式
\usepackage{here}
\usepackage{url}
\usepackage{lscape}
\usepackage{amsmath}
\usepackage{midpage}
\usepackage{exscale}

\usepackage{newtxtext,newtxmath} % Times系のフォントで英数字を表示する
\usepackage{xltxtra} % 游フォント好き...
\setCJKmainfont{游明朝 Regular}
\setCJKsansfont{游ゴシック Medium}
\setCJKmonofont{IPAGothic}

\bibliographystyle{IEEEtran} % 参考文献はIEEEの形式で入れる

\renewcommand{\figurename}{Fig.\,\,} % 図 1  じゃなくて Fig. 1 にしたかった
\renewcommand{\tablename}{Table\,\,} % 同上
\renewcommand{\thesection}{第\arabic{section}章} % ほんとは章じゃないけど

\begin{document}
    \section{序論\label{sec:序論}}

    \subsection{研究目的}
    本研究ではhogeを対象にhogeを導入した際のhogeをhoge分析することで快感を得る.先行研究はこれ\cite{Hoge2025}.

    \subsection{本論文の構成}
    本論文では,hoge導入に伴うhogeeとhoge排出量増加を考慮したhoge分析を行う.本論文の構成は次のとおりである.\\
    \ref{sec:序論}は,序論である.\\
    \ref{sec:手法}では,hoge分析に用いるhogeの推計手法について述べる.\\
    \ref{sec:手法2}では,hoge分析の手法について述べる.\\
    \ref{sec:結果と考察}では,hogeの推計およびhoge分析の結果と考察を述べる.\\
    \ref{sec:結論}では,結論を述べる.

    \newpage
    

\end{document}